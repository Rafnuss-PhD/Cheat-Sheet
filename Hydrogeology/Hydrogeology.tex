\documentclass[twocolumn]{article}
\usepackage{calc}
\usepackage[top=1cm,left=1cm,right=1cm,bottom=1cm]{geometry}
\usepackage{amsmath,amsthm,amsfonts,amssymb}
\usepackage{esint}
\usepackage{color,graphicx,overpic}
\usepackage{hyperref}
\usepackage{array} 
\usepackage{amstext}


\newcommand{\volume}{\mathop{\ooalign{\hfil$V$\hfil\cr\kern0.08em--\hfil\cr}}\nolimits}




\setlength{\parindent}{0pt}
\setlength{\parskip}{0pt plus 0.5ex}

%My Environments
% -----------------------------------------------------------------------

\begin{document}

\begin{center}
     \Large{Hydrogeologie} \\
\end{center}

\section{Porosity}
\begin{tabular}{@{}ll@{}}
$ V_t=V_s+V_p$ & Total volume\\
$ V_p=V_a+V_w$ & Pore volume\\
$ \Phi={V_p}/{V_t}$   & Total porosity \\
\\
$ S={V_w}/{V_p}$ & Water saturation\\
$ \theta ={V_w}/{V_t}$ & water content\\
$\theta=\Phi S$ &\\
$ S_e = \frac{\theta -\theta_s}{\theta_r-\theta_s}$ & Effective saturation\\
\\
$ r_h=\frac{A}{P} $  & Hydraulic radius \\
$ T= \left(\frac{L_a}{L}\right)^2 $ & Tortuosity \\
$ S_p= \frac{S}{m}$ & Specific surface area\\
\end{tabular}
with,
\begin{tabular}{@{}ll@{}}
$A$    & cross sectional area of flow  \\
$P$    & wetted perimeter \\
$F$ 	& Electrical formation factor\\
$S$ & Surface area\\
$m$ & masse\\
\end{tabular}

\begin{itemize}
	\item The porosity generated at the genesis of the rock (organogensis, clastic sedimentation) is called ``Primary'' and the one resulting from its geological history as the ``secondary'' (eg. tectonic or chemical process).
	\item The interconnected porosity consider only connected pore (electrical current and fluid can flow amongst them). Not very well define: depend on the scale, and gradual connection...
	\item Effective porosity is related to pore allowing free fluids. 
	\item Influence of grain properties: (1) $d_{50} \nearrow \Rightarrow\Phi \searrow$ (theoretical independant), (2) grain sorting $\nearrow \Rightarrow\Phi \nearrow$ and (3) rounder grain $\nearrow \Rightarrow \Phi \searrow$
	\item Porosity increase with depth (or pressure to be more precise). Many empirical equation exist (mainly exponential). For sandstone or sedimentary, it is roughly 5\% decrease in 100m and 40\% in 1km. 
	\item 
\end{itemize}




\section{Hydraulic Conductivity}
Hydraulic conductivity has first be discover with the very famous Darcy's law which found a propotionallity bewteen the discharge and the head difference:
$$Q=K\frac{\Delta P}{\Delta L}$$

It has since be derived from Navier-Stokes equation :
$$\boldsymbol{q}=-\frac{\kappa}{\mu}\left(\boldsymbol{\nabla} p -\rho \boldsymbol{g}\right)$$

Note $\kappa$ is the intrinsic permeability of the medium [L$^2$] while K is the hydraulic conductivity depending on the fluid [L/T] and can be related with:
$$ K=\frac{\kappa \rho_wg}{\mu} $$

This is for saturated medium otherwise, the saturated and relative K are used:
$$ K=K_rK_s$$

Allen Hazen (1893?) derived an empirical formula:
$$ K_s=C d_{10} $$
where $d_ {10}$ is the effective grain size for which 10\% is finer


Navier Stockes Equations
%http://en.wikipedia.org/wiki/Hagen%E2%80%93Poiseuille_flow_from_the_Navier%E2%80%93Stokes_equations
Hagen-Poiseuille Equation
Couette Flow

The Kozeny-Carman equation comes from Poiseuille Equation
$$v=\frac{\Delta(P)}{\eta L}\frac{\phi^{3}}{KS^{2}(1-\phi)^{2}}$$

And where:
$$\kappa = \frac{d^2}{180} \frac{\phi^3}{(1-\phi)^2}$$
 where d is Grain size.

\begin{tabular}{@{}ll@{}}
$a$  & Tube factor shape. \\
$F$ & Electrical formation factor\\
\end{tabular}

\section{Capillary Equations}
Young-Lapace Equation:
\[ \Delta p = -\gamma \nabla \dot \hat{n}=2\gamma H\]
Jurin's Law
\[ \psi=-\frac{2\gamma \cos \theta}{\rho_w g r}\]
\begin{tabular}{@{}ll@{}}
$\gamma$  & surface tension. \\
$H$ & mean curvature.\\
$\psi$ & capillary head.\\
$\theta$ & contact angle.\\
$r$ & radius of the tube
\end{tabular}

\section{Soil moisture characteristic}
Van Genuchten Equation (1980):
\[ S_e = \frac{1}{\left[ 1 + \alpha \psi^n \right]^m}\]
Mualem's Equation (1976)
\[ K_r= \sqrt{S_e} \left[ \frac{\int_0^{S_e}{dS_e / |\psi|}}{\int_0^1{dS_e/|\psi|}} \right] ^2 \]
\begin{tabular}{@{}ll@{}}
$\gamma$  & surface tension. \\
$H$ & mean curvature.\\
$\psi$ & capillary head.\\
$\theta$ & contact angle.\\
$r$ & radius of the tube
\end{tabular}


\section{Electrical Properties}

\subsection{General definition}
\[ \sigma^* = \frac{1}{\rho^*}=i \omega \varepsilon^* \]
\[ \sigma^* = |\sigma|e^{i\phi_\sigma} =\sigma' +i\sigma''\]
\[ \varepsilon = \varepsilon_r\varepsilon_0=(1+\chi)\varepsilon_0\]
\[ \kappa=\varepsilon_r=\frac{\varepsilon}{\varepsilon_0}\]

\subsection{Permittivity}
Refractive Index model (RI) for sat. and unsat.
\[ \sqrt{\varepsilon_{r,eff}} =n\sqrt{\varepsilon_{r,w}} +(1-n)\sqrt{\varepsilon_{r,s}}\] 
\[ \sqrt{\varepsilon_{r,eff}} =\theta\sqrt{\varepsilon_{r,w}} +(n-\theta)\sqrt{\varepsilon_{r,a}}+(1-n)\sqrt{\varepsilon_{r,s}}\]
 
\subsection{Electrical Conductivity}
Archie's empirical law (1942) for sat. and unsat.
\[ \sigma_{eff,sat} = \frac{\sigma_w}{F} (+\sigma_{surface})\]
\[\sigma_{surface}= \left\{
  \begin{array}{lr}
    \frac{BQ_v}{F} & \left\{
    				\begin{array}{l}
    					B=\alpha[1-\beta \exp(-\gamma\sigma_w)]\\
    					B=1.93m/(1+0.7/\sigma_m)
    				\end{array} \right. \\ 
    \frac{\Sigma_s S_p}{f} &
  \end{array}
\right.\]
\[ \sigma_{eff}=\sigma_{eff,sat}S^d\]
\[\sigma_{eff}= \left\{
  \begin{array}{lr}
    \sigma_{eff,sat}S^d &  \text{for } S>0.2\\ 
    \sigma_{w}n^mS^d & \text{if } \sigma_{surface}=0\\
    (\sigma_{w}+BQ_v/S)\frac{S^d}{F} & \text{(Waxman and Smits,1968)}\\
    \sigma_w \theta^m & \text{for/if } d=m,\sigma_{surface}=0\\
    \sigma_w \theta T_c(\theta)+\sigma_{surface} & Tc(\theta)=a\theta+b
  \end{array}
\right.\]
In a homogenous soil with a one-point current
\[V=\frac{\rho I}{2\pi r}\]
\subsection{Induced Polarisation}
\begin{tabular}{@{}ll@{}}
$\phi_\rho=\tan^{-1}\left(\rho''/\rho'\right)$  & CR phase. \\
$PFE=100\frac{\rho\left(\omega_1\right)-\rho\left(\omega_0\right)}{\rho\left(\omega_0\right)}$  & percent frequency effect \\
$M=\frac{V_s}{V_{p}}=\frac{1}{V_{p}\left(t_1-t_0\right)}\int_{t_0}^{t_1}V(t)dt$  & chargeability \\
\end{tabular}
\[\sigma^*=\frac{1}{F}\left(\sigma_w+BQ_v\right)+i\frac{\lambda Q_v}{Fn}\]
Relation KC type where $S_p$ is replace with a power law of $\sigma''$
\[ K_s=\frac{a}{FS_p^c}=\frac{a}{F(b\sigma''^p)^c}\]
Or Hazen type
\[ K_s=a(\sigma'')^b\]
\subsection{Coomplex Conductivtity}
Havriliak-Negami (HN)
\[ \varepsilon^*(\omega) = \left( \varepsilon_\infty + \frac{\sigma_{dc}}{i\omega}\right) \frac{\Delta \varepsilon}{ \left( 1+\left( i \omega \tau_0\right) ^{1-\alpha} \right) ^{\beta} } \]
Constnat Phase-Angle (CPA)
\[ \sigma^*(\omega)=(\sigma_{dc}+i\omega \varepsilon_\infty) + \sigma_0(\omega/\omega_0)^p e^{ip\pi/2} \]
The Bruggeman-Hanai-Sen (BHS) effective medium model
\[ \varepsilon^*=\varepsilon_w^*n^m\left(\frac{1-\varepsilon_s^*/\varepsilon_w^*}{1-\varepsilon_s^*/\varepsilon^*}\right)^m\]
\[\varepsilon_w^*=\varepsilon_{r,w}\varepsilon_0 + i\omega\sigma_w\]
\subsection{Legend}
\begin{tabular}{@{}ll@{}}
$\sigma^*,|\sigma|,\phi_\sigma,\sigma',\sigma''$	& conductivity (complex). \\
$\rho^*,|\rho|,\phi_\rho,\rho',\rho''$  			& complex resistivity. \\
$\varepsilon_{,r,0}^*$ 								& permittivity (relative, vaccum). \\
$\kappa =\varepsilon_r$ 							& dielectric constant\\
$\omega=2\pi f =1/\tau$  						& angular frequency, relaxation time \\
$\chi$ 											& electric susceptibility.\\
$F=T/n = n^{-m}$ 								& Electrical formation factor\\
$Q_v$ 											& cation exchange capacity (clay content)\\
$B$ 												& Ionic conductance of $Q_v$\\
$f$ 												& parameter of tortuosity\\
$m$ 												& cementation index (1.3<2.0)\\
$\Sigma_s$ 										& specific surface conductance\\
$f$ 												& parameter of tortuosity (F)\\
$d$ 												& saturation index (usally >m)\\
$\lambda$ 										& effective img. conductance\\
$\Delta \varepsilon = \varepsilon_{static} -\varepsilon_\infty$ & dielectric increment
\end{tabular}








\section{How to get to Poisson Law}
\subsection{Basic equation}
\begin{tabular}{@{}ll@{}}
$\boldsymbol{F_1}=\frac{q_1q_2}{4\pi\varepsilon_0} \frac{\boldsymbol{\hat{r}_{21}}}{ |\boldsymbol{r_{21}}|^2}, $ & Coulomb's law\\
$\mathbf{E}\equiv \frac{\mathbf{F}_{q}}{q}$ & Definition of electric field\\
$V = - \int_C \mathbf{E} \cdot \mathrm{d} \boldsymbol{\ell} \, $ & Electrical potential \\
$I = \frac{dq}{dt}$ & Intensity or current\\
$R = V/I$  & Ohm's law for resistance \\
$I=\int\mathbf{J}\cdot d\mathbf{A}$ & Current Flux\\
$\mathbf{J} = \sigma \mathbf{E}$ & Reformulation of Ohm's law \\
$R = \rho \frac{\ell}{A}$ & Pouillet's Law\\
$\Phi_E = \iint_S \mathbf{E} \cdot d\mathbf{S}$ & Definition of Electric Flux \\
$\lambda_q = \frac{d q}{d \ell}\,, \sigma_q = \frac{d q}{d S}\,, \rho_q =\frac{d q}{d \volume}$ & Density charges \\
$  \boldsymbol{D} = \varepsilon_0  \boldsymbol{E} +  \boldsymbol{P} $ & Electric displacement field \\
\end{tabular}

\subsection{From Coulomb to Poisson}
From Coulomb's law and the definition of Electric field
\[ \mathbf{E} = \frac{q}{4\pi\varepsilon_0 r^2}  \] 
And we get Gauss Law
\[ \Phi_E = \oiint_S \frac{q}{4\pi\varepsilon_0 r^2} \cdot \mathrm{d}\mathbf{S} = \frac{q}{\varepsilon_0}\]
And Using the Divergence Theorem (in/out = change inside), we can get the differential form of Gauss Law
\[ \iiint_V\left(\mathbf{\nabla}\cdot\mathbf{E}\right)\,dV=\oiint_S(\mathbf{E}\cdot\mathbf{n})\,dS .\]
\[\nabla \cdot \mathbf{E} = \frac{\rho_q}{\varepsilon_0}\]
An finally with $\mathbf{E} = -\nabla \cdot V$, we get Poisson's equation
 \[ {\nabla}^2 V = -\frac{\rho_q}{\varepsilon_0} \]


Continuity equation apply to charge density :
\[  \nabla \cdot \mathbf{J} = - \frac{\partial \rho_q}{ \partial t}  \]
And with $I = \frac{dq}{dt}$, $\mathbf{J} = \sigma \mathbf{E}$ and with linear charge density  $\rho_q =\frac{d q}{d V}$:
\[ \nabla \cdot (\sigma \mathbf{E}) = -\frac{\partial}{\partial t} \left(  \frac{dq}{d\volume}\right)\]
With a constant (over time) point charge $q$ at the origine $\rho_q(r) = q\delta(r)$
\[ \nabla \cdot (\sigma \nabla V) =-I \delta (r) \]


\section{Electromagnetism}
\subsection{Basic definition}
\begin{tabular}{@{}ll@{}}
$\mathbf{B} = \frac{\mu_0I}{4\pi}\int\frac{d\boldsymbol{\ell} \times \mathbf{\hat r}}{r^2}$ 	& Magnetic Field ( Biot-Savart law)\\
$\mathbf{F} = q \mathbf{v} \times \mathbf{B}$ 												& Lorentz force \\
$\mathbf{v}$ 																				& Particle's velocity \\
$\mu_0$																						& Magnetic constant\\
$\Phi_B = \iint\limits \mathbf{B} \cdot d \mathbf{S}$ 										& Magnetic flux\\
$\mathcal{E} = -{{d\Phi_\mathrm{B}} \over dt}$ 												& Electromotive force (EMF) (Faraday Law)\\
$\nabla \times \mathbf{E} = -\frac{\partial \mathbf{B}} {\partial t}$  						& Maxwell-Faraday equation \\
\end{tabular}

\subsection{Ampere law}
Ampere's law relates magnetic fields to electric currents that produce them:
\[ \oint_C \mathbf{B} \cdot \mathrm{d}\boldsymbol{\ell} = \mu_0 \iint_S \mathbf{J} \cdot \mathrm{d}\mathbf{S} = \mu_0I_\mathrm{enc}    \quad \text{or} \quad  \mathbf{\nabla} \times \mathbf{B} = \mu_0 \mathbf{J} \]

Current in a material is affected by magnetization (electrons remain bound to their respective atoms, but behave as if they were orbiting the nucleus in a particular direction, creating a microscopic current) and polarization (positive and negative bound charges can separate over atomic distances).
\[\mathbf{J} =\mathbf{J}_{\text{f}} + \mathbf{J}_{\text{M}} + \mathbf{J}_{\text{P}} \]

In order to take this effect to account and only count for the free current, the H magnetic field is introduced $\mathbf{H}\  \equiv \ \frac{\mathbf{B}}{\mu_0}-\mathbf{M}$ with ($M$: magnetization). And Ampere Law become :
\[\oint_C \mathbf{H} \cdot \mathrm{d}\boldsymbol{\ell} = \iint_S \mathbf{J}_\mathrm{f}\cdot \mathrm{d}\mathbf{S} = I_{\mathrm{f,enc}}     \quad  \text{or} \quad\mathbf{\nabla} \times \mathbf{H} = \mathbf{J}_{\text{f}} \]

\subsection{Issue with Ampere law and derivation of Maxwell–Ampère equation}
In vector calculus, $\nabla\cdot(\nabla\times\bold{B}) = 0 $ therefore according Ampere law $\nabla\cdot \bold{J} = 0$. But in genearale $\nabla\cdot \bold{J} = -\frac{\partial \rho}{\partial t} $. The Displacement current ($J_\mathbf{D}=\frac{\partial \mathbf{D}}{ \partial t}$) was added by Maxwell :
\[ \oint_C \mathbf{H} \cdot \mathrm{d}\boldsymbol{\ell} =  \iint_S \left( \mathbf{J}_{\mathrm{f}} + \frac{\partial }{\partial t}\mathbf{D} \right) \cdot \mathrm{d} \mathbf{S}    \quad \text{or} \quad \mathbf{\nabla} \times \mathbf{H} = \mathbf{J}_{\mathrm{f}}+\frac{\partial }{\partial t}\mathbf{D} \]
\[\mathbf{\nabla}\times \mathbf{B} = \left(\mu_0\mathbf{J}+\mu_0 \epsilon_0 \frac{\partial }{\partial t}\mathbf{E}\right) \]

\subsection{hello}
CSEM main equation result by combining Ampere-Maxwell, Ohm and Faraday laws into the damped wave equation
\[ \nabla ^2 \mathbf{B}  - \mu_0 \sigma \frac{\partial \mathbf{B}}{\partial t} - \mu_0 \epsilon \frac{\partial^2 \mathbf{B}}{\partial t^2}  = \mu_0 \nabla \times \mathbf{J}_S\] 
where







\section{Seismology}
\begin{tabular}{@{}ll@{}}
$A$ & Amplitude\\
$\lambda = 1/k$ & Wavelength (1/wave nb) (distence)\\
$T=1/f$ & Period (1/freq.)(time)\\
$V=f\lambda=1/Tk$&Propagation speed \\
$\frac{ \partial^2 u }{ \partial x^2 } = \frac{1}{V^2} \frac{ \partial^2 u }{ \partial  t^2 }$ & Wave equation\\
$V_p=\sqrt{\frac{K+4/3}{\rho}}$&P-wave velocity\\
$V_s=\sqrt{\frac{\mu}{\rho}}$&S-wave velocity\\
$K=\rho \frac{\mathrm d P}{\mathrm d \rho}$ & bulk modulus\\
$\rho$& density\\
$E$ & Module d'elasticity\\
$ \sigma = E \varepsilon$  & Module de traction \\
$\mu = G_s \ \varepsilon$ & Shear modulus\\
$ \nu = \frac{(l_0-l)/l_0}{(L-L_0)/L_0}$ & Coefficient poisson\\
$\mu = \frac {E}{2(1+\nu)}.$ &\\

\end{tabular}








\section{Geostat}
\subsection{Basic Definition}
\begin{tabular}{@{}ll@{}}
$ \Pr [ X \le x] = F_X(x) = \int_{-\infty}^x f_X(u) \, du $ & Prob. and cum. density function\\
$\operatorname{E}[X] = \int_{-\infty}^\infty x\,f(x)\,dx.$ & Expected value\\
$  \operatorname{Var}(X) = \operatorname{E}\left[(X - \mu)^2 \right] $ &Variance \\
$\operatorname{Cov}(X,Y) = \operatorname{E}{\big[(X - \operatorname{E}[X])(Y - \operatorname{E}[Y])\big]}$ & Covariance\\
$2\gamma(h)=\text{var}[Z(x) - Z(x+h)] $& Variogramme\\
$C(h)=\operatorname{cov}(Z(x),Z(x+h)).\,$ & Covariogramme\\
$C(0)=C(h)-\gamma(h)$ & Variance \\
\end{tabular}



\end{document}