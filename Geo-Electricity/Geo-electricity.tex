\documentclass[twocolumn]{article}
\usepackage{movie15}
\usepackage{animate}
\usepackage{calc}
\usepackage{ifthen}
\usepackage[margin=0.5in]{geometry}
\usepackage{amsmath,amsthm,amsfonts,amssymb}
\usepackage{amsfonts}
\usepackage{color,overpic}
\usepackage{hyperref}
\usepackage{array} 
\usepackage{amstext}
\usepackage{enumitem}
\usepackage{graphicx}
\usepackage{caption}
\usepackage{natbib}
\usepackage{framed}
\usepackage{float}

\newenvironment{Figure}
  {\par\medskip\noindent\minipage{\linewidth}}
  {\endminipage\par\medskip}
  
\numberwithin{equation}{section}

\graphicspath{ {./Illustration/} }

% Turn off header and footer
\pagestyle{empty}

\setlist{leftmargin=0cm}
\setlist{noitemsep}

\title{Electricity in the Soil}
\date{\vspace{-6ex}}

% -----------------------------------------------------------------------

\begin{document}
\maketitle




\newpage
\section{Electrical Law}	

	\subsection{Basic Definition}
\begin{tabular}{@{}ll@{}}
$\boldsymbol{F_1}=\frac{q_1q_2}{4\pi\varepsilon_0} \frac{\boldsymbol{\hat{r}_{21}}}{ |\boldsymbol{r_{21}}|^2}, $ & Coulomb's law\\
$\mathbf{E}\equiv \frac{\mathbf{F}_{q}}{q}$ & Definition of electric field\\
$V = - \int_C \mathbf{E} \cdot \mathrm{d} \boldsymbol{\ell} \, $ & Electrical potential \\
$I = \frac{dq}{dt}$ & Intensity or current\\
$R = V/I$  & Ohm's law for resistance \\
$I=\int\mathbf{J}\cdot d\mathbf{A}$ & Current Flux\\
$\mathbf{J} = \sigma \mathbf{E}$ & Reformulation of Ohm's law \\
$R = \rho \frac{\ell}{A}$ & Pouillet's Law\\
$\Phi_E = \iint_S \mathbf{E} \cdot d\mathbf{S}$ & Definition of Electric Flux \\
$\lambda_q = \frac{d q}{d \ell}\,, \sigma_q = \frac{d q}{d S}\,, \rho_q =\frac{d q}{d \volume}$ & Density charges \\
$  \boldsymbol{D} = \varepsilon_0  \boldsymbol{E} +  \boldsymbol{P} $ & Electric displacement field \\
\end{tabular}

	\subsection{From Coulomb to ...}
From Coulomb's law and the definition of Electric field
\[ \mathbf{E} = \frac{q}{4\pi\varepsilon_0 r^2}  \] 
And we get Gauss Law
\[ \Phi_E = \oiint_S \frac{q}{4\pi\varepsilon_0 r^2} \cdot \mathrm{d}\mathbf{S} = \frac{q}{\varepsilon_0}\]
And Using the Divergence Theorem (in/out = change inside), we can get the differential form of Gauss Law
\[ \iiint_V\left(\mathbf{\nabla}\cdot\mathbf{E}\right)\,dV=\oiint_S(\mathbf{E}\cdot\mathbf{n})\,dS .\]
\[\nabla \cdot \mathbf{E} = \frac{\rho_q}{\varepsilon_0}\]
An finally with $\mathbf{E} = -\nabla \cdot V$, we get Poisson's equation
 \[ {\nabla}^2 V = -\frac{\rho_q}{\varepsilon_0} \]


Continuity equation apply to charge density :
\[  \nabla \cdot \mathbf{J} = - \frac{\partial \rho_q}{ \partial t}  \]
And with $I = \frac{dq}{dt}$, $\mathbf{J} = \sigma \mathbf{E}$ and with linear charge density  $\rho_q =\frac{d q}{d V}$:
\[ \nabla \cdot (\sigma \mathbf{E}) = -\frac{\partial}{\partial t} \left(  \frac{dq}{d\volume}\right)\]
With a constant (over time) point charge $q$ at the origine $\rho_q(r) = q\delta(r)$
\[ \nabla \cdot (\sigma \nabla V) =-I \delta (r) \]









\bibliographystyle{apalike}
\bibliography{citations}	
	

\end{document}